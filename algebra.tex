# Algebra

# Gruppi

- $S$ insieme e $m: S \times S \rightarrow S$
  - $(S, m)$ **semigruppo** $\iff$ vale la **proprietà associativa** in $m$ su $S$
    - $m(x, m(y, z))=m(m(x, y),z) \quad \forall x, y, z \in S$
  - $(S, m)$ **monoide** $\iff$ è un semigruppo in cui **esiste l'elemento neutro** rispetto a $m$
    - $\exists e  \mid m(x, e) = m(e, x) = x \quad \forall x \in S$
    - se esiste, $e$ è unico
        - per assurdo, $\exists e_1, e_2 \mid e_1 \neq e_2$ elementi neutri, allora \( \left.\begin{array}{l}m\left(x, e_{1}\right)=m\left(e_{1}, x\right)=x \\ m\left(x, e_{2}\right)=m\left(e_{2}, x\right)=x\end{array}\right\} \Rightarrow m\left(e_{1}, x\right)=m\left(e_{2}, x\right) \implies e_1=e_2\) necessariamente

  - $(S, m)$ **gruppo** $\iff$ è un monoide in cui **esiste l'inverso** per ogni elemento di $S$
    - $\exists x^{-1} \mid m(x, x^{-1}) =m(x^{-1}, x) =e \quad \forall x \in S$
    - se esiste, $x^{-1}$ è unico
      - ⚠️ **MANCA DIMOSTRAZIONE**
  - $(S, m)$ **gruppo abeliano** $\iff$ è un gruppo in cui vale la **proprietà commutativa** in $m$ su $S$
    - $m(x, y) = m(y, x) \quad \forall x, y \in S$

## Esempi

- $X, Y$ insiemi, $Y^X = \{f \mid f:X \rightarrow Y\}$
  - $X, Y$ finiti $\Rightarrow \left| Y^X \right| = \left| Y \right| ^ {|X|}$
      - ⚠️ **MANCA DIMOSTRAZIONE**
  - $(X^X, \circ)$ è **monoide**
    - \( (f \circ g) \circ h=f \circ(g \circ h) \)
    - $\forall X, \ \exists \textrm{id}_X \mid \textrm{id}_X : X \rightarrow X : x \rightarrow x$, che costituisce dunque l'elemento neutro, mappando ogni elemento in sé stesso
  - $S_X = \{f \mid f : X \rightarrow Y \ \textrm{biiettiva}\}$ è detto **gruppo simmetrico di $X$**
    - $|S_X|$ = $|X|!$
    - $(S_X, \circ)$ è un **gruppo**, non commutativo se $|X| \ge 3$

# Anelli

- $A$ insieme
- $+: A \times A \implies A$
- \( *: A \times A \implies A \)

- \( (A, +, *) \) anello \(\iff  \)
  - $(A, +)$ **gruppo abeliano**
  - $(A, *)$ **monoide**
  - vale la **proprietà distributiva** della forma \(a*(b + c) = a * b + a * c\)

- \( a * b=b * a \quad \forall a, b \in A \implies  \rm (A, *, +)\) è un **anello commutativo**

- \(\exists x^{-1} \quad \forall x \in A \mid x * x^{-1}=x^{-1} * x=e  \implies  (A, +, *)\) è un **campo**

## Esempi

- $(\mathbb{Z}, +, \cdot)$ è un **anello commutativo**
- $(\mathbb{C}, +, \cdot)$ è un **campo**
- ⚠️ **MANCA DIMOSTRAZIONE** polinomi a coefficienti in $A$

# Numeri complessi

- \(\mathbb{C}=\left\{a+i b \mid a, b \in \mathbb{R}, \  i \mid i^{2}=-1\right\} \)

- \( z \in \mathbb{C} \implies\left\{\begin{array}{l}a:=\operatorname{Re}(z) \\ b:=\operatorname{Im}(z)\end{array}\right. \)

- \( \mathbb{R} \subset \mathbb{C} \)

- \(
\left\{\begin{array}{l}
z=a+i b \\
w=c+i d
\end{array} \implies z+w=(a c-b d)+i(ad+ bc)\right.
\)

- \( z=a+i b \implies \bar{z}:=a-i b \)
  - \( \overline{z}+\overline{w}=\overline{z+w} \)
  - \( \overline{z} \cdot \overline{w}=\overline{z \cdot w} \)
- \( |z|=\sqrt{a^{2}+b^{2}} \)
  - \( z \in \mathbb{C}, z \neq 0 \implies z=|z| e^{i \theta} \) dove \( e^{i \theta}=\cos \theta+i \sin \theta \) è detta **formula di Eulero**
  - \( \arg(z) := \left\{\begin{array}{l}\cos \theta=\frac{a}{|z|} \\ \sin \theta=\frac{b}{|z|}\end{array}\right. \implies \exists ! \ 0 \leq \theta \le 2 \pi \)
    - \( \textrm{Arg}(z) :=\theta \)
    - \(\arg(z) \subset \mathbb{R}\) è l'insieme delle soluzioni del sistema, mentre $\textrm{Arg}(z)$ è la **soluzione principale**

- \(
\left.\begin{array}{l}
z \cdot \bar{z}=(a+i b)(a-i b)=a^{2}-(i b)^{2} \\
i^{2}=-1
\end{array} \right \} \implies a^{2}-i^{2} b^{2}=a^{2}+b^{2}=|z|^{2} 
\)

  - \( z \cdot \bar{z}=|z|^{2} \implies z=\dfrac{|z|^{2}}{\bar{z}} \implies z^{-1}=\dfrac{z}{|z|^{2}} = \dfrac{a}{a^2+b^2}- i \dfrac{b}{a^2+b^2}\)
  - \( \dfrac{z \cdot \bar{z}}{|z|^{2}}=1 \implies \mathbb{C}\) ammette inversi moltiplicativi \( \implies (\mathbb{C}, +, *) \) è un **campo**

- \(|z \cdot w|=|z||w|, \ \arg(z\cdot w)=\arg(z) + \arg(w)\)
- \(|\overline{w}|=|w|, \ \arg(\overline{w})=-\arg(w)\)
- \(|w^{-1}|={|w|}^{-1},\  \arg(w^{-1})=-\arg(w)\)
- \(\left|\dfrac{z}{w}\right|=\dfrac{|z|}{|w|}, \ \arg\left(\dfrac{z}{w}\right)=\arg(z) - \arg(w)\)
- **Formula di de Moivre**
  - \( z^{n}=r^{n} e^{i n \theta}, \ \arg \left(z^{n}\right)=n \arg (z) \)

****

## Teorema fondamentale dell'algebra

Data un'equazione \( a_{0}+a_{1} x+a_{2} x^{2}+\cdots+a_{n} x^{n}=0 \), con \( a_{0}, a_{1}, a_{2}, \ldots, a_{n} \in \mathbb{C}, \ n \geq 1, a_n \neq 0 \quad \implies \quad \exists x \in \mathbb{C}\) 

****

# Relazioni

- dato un insieme \(S\), allora \(R := R \mid R \subseteq S \times S\)
- \(R\) è una **relazione di equivalenza** \(\iff \)
  - **riflessiva**: $R$ riflessiva $\iff xRx \quad \forall x \in S$
  - **simmetrica**: $R$ simmetrica $\iff xRy \implies yRx \quad \forall x, y \in S$
  - **transitiva**: $R$ transitiva $\iff xRy, yRz \implies xRz \quad \forall x, y, z \in S$
- $R$ è un **ordine parziale** $\iff$
  - $R$ **riflessiva**, **transitiva** e **antisimmetrica**
    - $R$ **antisimmetrica** $\iff xRy, yRx \implies x=y \quad \forall x, y \in S$
- $R$ **ordine totale** $\iff$ 
  - $R$ ordine parziale in cui vale la **totalità**
    - $R$ **totale** $\iff xRy \lor yRx \quad \forall x, y \in S$

## Esempi

  - $\forall X, \ A, B \subset P(X)$,  $A \subset B$ è **ordine parziale** su $P(X)$
    - ⚠️ **MANCA DIMOSTRAZIONE**
  - $m, n \in \mathbb{N}, \ m \mid n$ ("m **divide** n") $\iff \exists p \in \mathbb{N} \mid mp = n$
    - è **ordine parziale**
      - *riflessività*: \( \forall x \in \mathbb{N}, x\mid x \Rightarrow \exists p \in \mathbb{N} \mid  x p=x \implies p = 1 \in \mathbb{N}\)
      - *transitività*: \( \forall d, m, m \in \mathbb{N}, \  d \mid m \wedge m| m \implies d \mid m \)
        -  \( \left.\begin{array}{l}d \mid m \Rightarrow \exists p_{1} \in \mathbb{N}\mid d p_{1}=m \\ m\mid m \Rightarrow \exists p_{2} \in \mathbb{N}\mid m p_{2}=n\end{array}\right\} \Rightarrow d p_{1} p_{2}=n \Rightarrow d \mid n \) poiché \(p_1 \in \mathbb{N} \land p1 \in \mathbb{N} \implies p_1 p_2 \in \mathbb{N}\)
      -  *antisimmetria*: \( \forall m, m \in \mathbb{N}, \ m\mid m \wedge m\mid m \implies m=n \)
        -  \( \left.\begin{array}{l}m\mid n \Rightarrow \exists p_{1} \in \mathbb{N}\mid m p_{1}=n \\ n\mid m \Rightarrow \exists p_{2} \in \mathbb{N}\mid n p_{2}=m\end{array}\right\} \Rightarrow m p_{2} p_{2}=m \implies p_1 p_2 = 1 \implies p_1 = p_2 = 1\) perché \(p_1, p_2 \in \mathbb{N}\), quindi \(np_2=m \land p_2 = 1 \implies n =m\)

- \( a, b \in \mathbb{Z}, \  a \equiv b \ (\bmod n) \iff m \mid b-a \) detta **congruenza modulo $n$**
  - è una **relazione di equivalenza**
    - _riflessività_: $\forall a \in \mathbb{Z}, \ a \equiv a \ (\bmod n)  \implies n \mid a - a \implies n \mid 0$, e $n \mid 0 \implies \exists p \in \mathbb{Z} \mid n \cdot p = 0 \implies p = 0 \in \mathbb{Z}$
    - _simmetria_: \(\forall a, b \in \mathbb{Z}, \ a \equiv b \ (\bmod n) \implies b \equiv a \ (\bmod n)\)
      - $a \equiv b \ (\bmod n) \implies n \mid b - a \implies \exists p_1 \in \mathbb{Z} \mid n \cdot p_1 = b - a$
      - $b \equiv a \ (\bmod n) \implies n \mid a - b \implies \exists p_2 \in \mathbb{Z} \mid n \cdot p_2 = b - a$
       - \( \left.\begin{array}{l}n p_{1}=b-a \implies b=n p_{1}+a \\ n p_{2}=a-b\end{array}\right\} \implies np_2 = a - np_1 - a = -np_1 \implies n(p_2 + p_1)=0 \)
       - \( n \neq 0 \), quindi \(p_{2}+p_{1}=0 \implies p_{2}=-p_{1} \)
       - per definizione di $p_2$, $np_2 = b - a \implies n (-p_1) = b - a \implies (-1) \cdot n p_1 = b - a \implies n p_1 = a -b \implies n \mid b - a$

    - _transitivtà_: \(\forall a, b, c \in \mathbb{Z}, \ a \equiv b \ (\bmod n), b \equiv c \ (\bmod n) \implies a \equiv c \ (\bmod n)\)
      - $a \equiv b \ (\bmod n) \implies n \mid b - a \implies \exists p_1 \in \mathbb{Z} \mid n \cdot p_1 = b - a$
      - $b \equiv c \ (\bmod n) \implies n \mid c - b \implies \exists p_2 \in \mathbb{Z} \mid n \cdot p_2 = b - a$
      - \( \left.\begin{array}{l}n p_{1}=b-a \implies b=n p_{1}+a \\ n p_{2}=c-b\end{array}\right\} \implies np_2 = c - np_2 - a \implies np_2 + np_1 = c - a \implies n(p_2 + p_1)=c -a \)
      - $p_{1}, p_{2} \in \mathbb{Z} \implies p_{1}+p_{2} \in \mathbb{Z} \implies \exists p_1 + p_2 \in \mathbb{Z} \mid n(p_1 + p_2) = c - a \implies n \mid c - a$ per definizione
- $x \sim y \iff x^{-1}y \in H$
  - è una **relazione di equivalenza**
    - ⚠️ **MANCA DIMOSTRAZIONE**

**** 

# Teorema della divisione euclidea con il resto

$$
m, n \in \mathbb{Z}, n>0 \implies \exists !  \ q, r \in \mathbb{Z} \mid m=n q+r, \ 0 \leq r<n
$$

****

# Sottogruppi

- $H \subset G$ **sottogruppo** di un gruppo $(G, *)\iff$
    - $\exists e \in H \mid e$ è l'**elemento neutro**
    - $H$ è **chiuso rispetto all'operazione $*$**
        - $\forall x, y \in H, \ x * y \in H$
    -  $H$ è **chiuso rispetto agli inversi**
        -  $\forall x \in H, \ \exists x^{-1} \in H$
-  \( (\mathbb{Z},+) \subset(\mathbb{Q},+) \subset(\mathbb{R},+) \subset(\mathbb{C},+) \) tutti sottogruppi

****

# Classi di equivalenza

- $[x] := \{y \in S \mid x \sim y\}$
  - $x \in [x] \quad \forall x \in S$
    - $x \sim x \quad \forall x \in S$ per definizione
  - $\forall x, y \in S, \ [x] = [y] \iff x \sim y \ \lor \ [x] \cap [y] = \varnothing \iff x \nsim y$, quindi **due classi di equivalenza o coincidono, o non si intersecano**
      - se \( \left.x \sim y, \exists z \in[x] \Rightarrow \ \begin{array}{ll}z  \sim x \\ x \sim y\end{array}\right\} \ z \sim y \ \) per transitività, quindi $z \in [y]$
      - se \( \left.y \sim x, \exists z \in[y] \Rightarrow \ \begin{array}{l}z \sim y \\ y \sim x\end{array}\right\} \ z \sim x \ \) per transitività, quindi $z \in [x]$
      - quindi \( \forall z \in[x], \ x \sim y \implies z \in[y] \) e $\forall z \in  [y], \ y \sim x \implies z \in [x]$, quindi $[x] = [y]$ necessariamente
      - se $x \nsim y$, e per assurdo \( [x] \cap[y] \neq \varnothing \) allora \( \exists z \mid z \in[x] \wedge z \in[y] \Rightarrow z \sim x \wedge z \sim y \implies x \sim y \) per transitività

- \(S/ \sim \ = \left\{ [x] \mid x \in S\right\}\) è l'insieme di tutte le classi di equivalenza, detto **insieme quoziente**
  - presa come relazione di equivalenza la congruenza modulo $n$, si definisce $\mathbb{Z}_n = \{ [0], [1], \ldots,[n - 1]\} \implies \mid \mathbb{Z}_n \mid = n$, in cui ogni elemento è la classe di equivalenza di ogni intero fino ad $n - 1$, e $[x] = \{y \in \mathbb{Z} \mid y \equiv x  \ (\bmod n )\}$
  - \( \exists ! \ q, r \in \mathbb{Z} \mid m=n q+r \quad \forall m, n \in \mathbb{Z} \) per il teorema della divisione euclidea con il resto, dunque \( \exists q \mid  m=n q+r \implies n q=m-r \implies n \mid m-r \implies \exists q \mid m \equiv r \ (\bmod n) \implies [x] \in \mathbb[Z]_n, [x] \neq \varnothing \quad \forall n \in \mathbb{Z}\)

## Teorema di Lagrange (teoria dei gruppi)

- $xH = \{xh \mid h \in H\}$ dove $H \subset G$ e $x \in G$, è detta **classe laterale sinistra** di $H$ in $G$
  - quando $G$ è finito, $| xH | = | H |$ perché per ogni elemento $x$ che genera $xH$, $xH$ è l'insieme dei prodotti di $x$ con ogni elemento di $H$
    - $H \rightarrow xH$ è biunivoca $\forall x \in G$
  - $G / H = \{xH \mid x \in G\}$ è l'insieme delle classi laterali sinistre, e poiché sono disgiunte a due a due, e la loro unione equivale a $G$, allora ogni $xH$ è una **partizione** di $G$
  - $|G| = |H| \cdot [G : H]$ è il **teorema di Lagrange**
    - $|G|$ è la cardinalità di $G$
    - $|H|$ è la cardinalità di $H$, che equivale a $|xH| \ \forall x \in G$
    - $[G : H]$ è la cardinalità di $|G / H|$, ovvero il numero di classi laterali sinistre

****

# Ideali

- $(A, +, *)$ anello commutativo
- $I \subset A$ **ideale** $\iff$
  - $(I, +) \subset (A, +)$ è un **sottogruppo**
  - $\forall x \in I, a \in A \implies ax \in I \implies A \cdot I \subset I$
- nel caso in cui $(A, +, *)$ non sia commutativo, basta aggiungere che $I \cdot A \subset I$
- $I \subset \mathbb{Z}$ ideale $\implies$ \( \exists ! \ d \geq 0 \mid I=I(d):=\{x d \mid x \in \mathbb{Z}\} \), dove $I(d)$ è un ideale, detto **ideale principale generato da $d$**
  - *esistenza*
    - $d:=\min(I \cap \mathbb{Z}_{\gt 0})$
      - se $I = \{0\} \implies I = I(0)$, altrimenti \( I \cap \mathbb{Z}_{>0} \neq \varnothing \)
        - \( \forall x \in I \mid x < 0 \implies (-x)>0 \), e $(-x) \in I$ per definizione di $I$, quindi anche se ho un numero negativo, posso considerare il suo opposto per la dimostrazione
    - \( I(d)=I \implies I(d) \subset I \wedge I \subset I(d) \)
      - $I(d) \subset I$
        - $\forall x \in I(d), \exists y \in \mathbb{Z} \mid x = dy$ per definizione
        - $d \in I$ per definizione, quindi $dy \in I \implies x \in I \implies I(d) \subset I$ in quanto $I \subset \mathbb{Z}$ ideale, e dunque $I \cdot \mathbb{Z} \subset I$ (poiché $\mathbb{Z}$ è anello commutativo)
      - $I \subset I(d)$
        - \( \forall x \in I, \exists ! q,r \in \mathbb{Z} \mid x=d q+r, \quad 0 \leq r<d \), per il teorema della divisione euclidea con il resto, e $d\neq 0$ per ipotesi
          - $r = 0 \implies x = dq \implies x \in I(d)$ per definizione, dunque $I \subset I(d)$
          - se, per assurdo, $r \neq 0$
            - $x \in I$ per ipotesi, $dq \in I(d) \implies dq \in I$ per dimostrazione precedente, quindi $x = dq + r \implies r = x - dq \in I$, ma poiché $r \neq 0$ per ipotesi, allora $r \in I \cap \mathbb{Z}_{\gt 0}$
            - per definizione, $0 \le r \lt d$, ma $d:=\min(I \cap \mathbb{Z}_{\gt 0})$, quindi il minimo numero che $d$ può assumere è $1$, e poiché $r < d \implies r = 0$ necessariamente

  - *unicità*
    - \( I(d)=I(-d)\), quindi l'unicità deriva dal fatto che $d:=\min(I \cap \mathbb{Z}_{\gt 0})$, e dunque nella dimostrazione è preso $d$ positivo, ma vale il ragionamento analogo per $d < 0$ considerando $I(-d)$
      - \( I(a)=I(b) \iff a=\pm b \quad \forall a, b \in \mathbb{Z} \mid a \neq b\)
        - \( a=\pm b \implies I(a)=I(b) \)
          - $a = b \implies I(a)$ e $I(b)$ coincidono
          - $a = -b$ allora $I(-b) = \{ k(-b) \mid k \in \mathbb{Z}\} = \{(-k)b \mid (-k) \in \mathbb{Z}\} = I(b)$, e $k, -k \in \mathbb{Z} \quad \forall k \in \mathbb{Z}$
        - $I(a) = I(b) \implies a = \pm b$
          - $I(a) = I(b) \implies a \in I(b)$ e $b \in I(a) \implies \exists p, q \in \mathbb{Z} \mid a = pb \wedge b = qa$, di conseguenza $b = q (pb) \implies b = (qp)b \implies pq = 1 \implies p = q = 1 \lor p = q = -1 \implies a = \pm b$
  - $I(d)$ ideale
    - ⚠️ **MANCA DIMOSTRAZIONE**
    - più in generale, $I(a_1, \ldots, a_n) = \{ a_1b_1 + \ldots a_nb_n \mid b_1, \ldots b_n \in A\}$ è l'**ideale di $A$ generato dagli $a_1, \ldots, a_n \in A$**
      - $I$ induce una relazione di equivalenza su $A$ detta **congruenza modulo $I$**
        - $a \equiv b \ (\bmod I) \iff b - a \in I$

## Massimo comun divisore

- \( \forall a_{1}, \ldots a_{n} \in \mathbb{Z}, \quad \exists I\left(a_{1}, \ldots a_{n}\right) \mid \exists ! d \geq 0  : I\left(a_{1}, \ldots a_{n}\right)=I(d),   \ \ d:=\textrm{MCD}(a_{1}, \ldots a_{n}) \)
    - ⚠️ **MANCA DIMOSTRAZIONE**
        - $\forall x \in I(a_1, \cdots, a_n), \ d \mid x$, dunque $d$ è *divisore comune*
        - $d$ è il _massimo tra i divisori comuni_
  - **identità di Bézout**
    - \( \exists x, y \in \mathbb{Z} \mid a x+b y=d \quad \forall a, b \in \mathbb{Z} \)
 
## Operazioni sugli ideali

- su $I, J \subset A$ ideali in $A$ anello commutativo, è possibile definire $I + J$, $I \cap J$ e $I \cdot J$
  - $I + J = \{i + j \mid i \in I, j \in J\}$
    - $I + J$ **sottogruppo**
      - \( 0 \in I, 0 \in J, 0+0=0 \implies 0 \in I + J \) per definizione
      - la chiusura rispetto a $+$, implica che $\forall i_1, i_2 \in I, j_1, j_2 \in J \quad (i_1 + j_1) + (i_2 + j_2) \in I + J$, e poiché $(i_1 + j_1) + (i_2 + j_2) = (i_1 + i_2) + (j_1 + j_2)$, e $i_1 + i_2 \in I, j_1 + j_2 \in J$ , allora per definizione di $I + J$, $(i_1 + i_2) + (j_1 + j_2) \in I + J$
      - $\forall i \in I, j \in J \quad i + j \in I + J$, l'opposto rispetto a $+$ di $i + j$ è $- (i + j) = (-i) + (-j)$, e $-i \in I, -j \in J \quad \forall i \in I, j \in J \implies (-i) + (-j) \in I + J$ per definizione
    - $A \cdot I \subset I \implies \forall a \in A, i \in I, j \in J \quad a(i + j) \in I + J$
      - $i + j \in I + J$ per definizione, e $a(i + j) = ai + aj$, e $ai \in I, aj \in J$ per definizione, quindi $ai + aj \in I + J$ per definizione
  - $I \cap J = \{x \in I \land x \in J\}$
    - ⚠️ **MANCA DIMOSTRAZIONE**
  - $I \cdot J = \{i_1 j_1 + \ldots + i_k j_k \mid k \ge \ 1, i_1 , \ldots , i_k \in I, j_1 , \ldots , j_k \in J \}$
    - ⚠️ **MANCA DIMOSTRAZIONE**
- $\mathbb{Z}$ è un **anello ad ideali principali**
  - $\forall a, b \in \mathbb{Z} \quad I(a) + I(b) = I(d), \quad d:= \textrm{MCD}(a, b)$
    - \( I(a)+I(b)=\{i+j \mid i \in I(a), J \in I(b) \} \), ma $i \in I(a) \implies \exists x \in \mathbb{Z} \mid i = ax$ e $j \in I(b) \implies \exists y \in \mathbb{Z} \mid j = by$, quindi $i + j = ax + by \implies$ \( I(a)+I(b)=\{a x+b y \mid x, y \in \mathbb{Z}\} = I(a, b) \) per definizione, e per l'identità di Bézout, $\exists x, y \in \mathbb{Z} \mid ax + by = d := \textrm{MCD}(a, b)$, e per teoremi precedenti, $I(a, b) = I(d)$
  - \( \forall a, b \in \mathbb{Z} \quad I(a) \cdot I(b)=I(a \cdot b) \)
    - \( x \in I(a) \cdot I(b) \implies x \in I(a \cdot b) \)
      - $x \in I(a) \cdot I(b) \implies x = i_1 j_1 + \ldots + i_k j_k$ con $i_1 , \ldots , i_k \in I(a), j_1 , \ldots , j_k \in I(b)$, ma per definizione, $i \in I(a) \implies \exists x \in \mathbb{Z} \mid i = ax$, e dunque $i_1, \ldots, i_k = ax_1, \ldots, ax_k$ con $x_1, \ldots, x_k \in \mathbb{Z}$, e analogamente $j_1, \ldots, j_k = by_1, \ldots, by_k$ con $y_1, \ldots, y_k \in \mathbb{Z}$
      - segue che $x = (ax_1)(by_1),+\ldots+ (ax_k)(by_k) = ab\cdot(x_1y_1+ \ldots+ x_ky_k)$, e poiché $(x_1y_1+ \ldots+ x_ky_k) \in \mathbb{Z}$, per definizione segue che $x \in I(a\cdot b)$
    - \( x \in I(a \cdot b)  \implies x \in I(a) \cdot I(b)\)
      - $x \in I(a \cdot b) \implies \exists k \in \mathbb{Z} \mid x = ab \cdot k$, ma \( x=a b k \implies\left\{\begin{array}{l}x=a \cdot b k \implies \exists bk \in \mathbb{Z} \mid x = a \cdot bk \implies x \in I(a) \\ x=b \cdot a k \implies \exists ak \in \mathbb{Z} \mid x = b \cdot ak \implies x \in I(b)\end{array}\right. \) **INCOMPLETA**
   
## Minimo comune multiplo

- $\displaystyle{\forall a_{1}, \ldots, a_{n} \in \mathbb{Z}  \quad \exists ! m  \in \mathbb{N} \mid m:= \textrm{mcm}(a_1, \ldots, a_n) : I(m) = I(a_1) \cap \ldots \cap I(a_n) = \bigcap_{i=1}^{n}{I(a_i)}}$

****

# Invertibili e divisori dello $0$

- $(A, +, \cdot)$ anello commutativo
  - $a \in A$ è detto **invertibile** \( \iff \exists a^{-1} \in A \mid a \cdot a^{-1}=e \)
    - $A^* := \{a \in A \mid a \ \textrm{invertibile}\} \subset A$
    - $(A^*, \cdot)$ è un **sottogruppo** di $(A, \cdot)$
      - $1^{-1} = 1 \implies 1$ invertibile $\implies 1 \in A^*$ per definizione di $A^* \implies \exists e \in A^*$
      - \( \forall x, y \in A^{*} \quad x \cdot y \in A^{*} \)
      - \( \forall x \in A^{*} \quad \exists x^{-1} \) per definizione di $A^*$, ma poiché $x^{-1}$ è inverso di $x$, allora $x^{-1} \in A^*$ per definizione
    - $(A^*, \cdot)$ è un **gruppo**
      - $(xy)z = x(yz)$
      - $\exists e$ ed è $1 \in A^*$
      - \( \forall x \in A^{*} \quad \exists x^{-1}\) per definizione

  - $a \in A$ è detto **divisore dello $0$** \( \iff \exists b \in A, b \neq 0 \mid a \cdot b=0 \)
    - $A$ è detto **dominio di integrità** $\iff \nexists x \mid x \textrm{  divisore dello 0}$ oltre a $x = 0$
    - $A$ è dominio di integrità $\iff$ in $A$ vale la legge di annullamento del prodotto
      - un divisore dello $0$ non è invertibile

*****

# Insiemi quoziente $\mathbb{Z}_n$

- $\mathbb{Z}_n$ dominio $\iff$ $n$ primo
      - ⚠️ **MANCA DIMOSTRAZIONE**
- $\forall [x] \in \mathbb{Z}_n, \ \textrm{MCD}(x, n) = 1 \iff [x] \in \mathbb{Z}^*_n$
      - ⚠️ **MANCA DIMOSTRAZIONE**
    - $p$ primo $\implies \mathbb{Z}_p^* = \{[x] \in \mathbb{Z}_p \mid  0 \lt x \lt p\} = \mathbb{Z}_p - \{0\}$
        - $p$ primo $\implies$ ogni numero è coprimo con $p$
        - $\nexists x \mid [0]$ invertibile
        - $[p] \notin \mathbb{Z}_p$ per definizione di $\mathbb{Z}_p$
        - $p$ primo $\implies \mathbb{Z}_p$ campo

****

# Teorema fondamentale dell'aritmetica

- $\forall a, b \in \mathbb{N} \quad \textrm{mcm}(a, b) \cdot \textrm{MCD}(a, b) = a \cdot b$
  - $a = 0 \lor b = 0 \lor a, b = 0 \implies \textrm{mcm}(a, b) = 0$ **INCOMPLETA**
  - $a, b \gt 0$
    - $\mathbb{P} = \{p \in \mathbb{N} \mid p \textrm{ primo}\}$
    - $\forall n \in \mathbb{N} - \{0\} \quad \exists ! n_2, n_3, n_5, \ldots, n_p \in \mathbb{N} \mid p \in \mathbb{P} : n = 2^{n_2} \cdot 3 ^ {n_3} \cdot \ldots \cdot p ^ {n_p}$
      - $p \nmid n \implies n_p = 0 \implies p ^  {n_p} = 1$, dunque non influisce nella produttoria
    - $\displaystyle{n = \prod_{p \in \mathbb{P}}^{} p ^{n_p}}$, quindi possiamo riscrivere anche $a$ e $b$ tramite i loro fattori primi
      - \(\displaystyle{a=\prod_{p \in \mathbb{P}} p^{a_{p}}} \) e \( \displaystyle{b=\prod_{p \in \mathbb{P}} p^{b_{p}} }\)
    - $d:= \textrm{MCD}(a, b)$ e $m:=\textrm{mcm}(a, b)$
      - per definizione di $d$ ed $m$, e attraverso le regole che permettono di trovarli tramite le fattorizzazioni di $a$ e $b$, è possibile riscrivere $d$ ed $m$ come $\displaystyle{d = \prod_{p \in \mathbb{P}} p^{\min(a_p, b_p)}}$ e $\displaystyle{m = \prod_{p \in \mathbb{P}} p^{\max(a_p, b_p)}}$
      - $d \cdot m =\displaystyle{\prod_{p \in \mathbb{P}} p^{\min(a_p, b_p)}} \cdot \displaystyle{\prod_{p \in \mathbb{P}} p^{\max(a_p, b_p)}} = \displaystyle{\prod_{p \in \mathbb{P}} p^{\min(a_p, b_p) + \max(a_p, b_p)}}$
    - $\forall a, b \in \mathbb{N} \quad a + b = \min(a, b) + \max(a, b)$
      - $a = \min(a, b) \implies \max(a, b) = b$, e viceversa
    - $d \cdot m = \displaystyle{\prod_{p \in \mathbb{P} }p ^{a_p + b_p}} = \displaystyle{\prod_{p \in \mathbb{P}} p^{a_p}} \cdot \displaystyle{\prod_{p \in \mathbb{P}} p^{b_p}} = a \cdot b$

****

# Teorema cinese dei resti

## Lemma 1

## Lemma 2

## Teorema

- $\forall a_1, \ldots, a_n \ge 2 \in \mathbb{Z} \mid \textrm{MCD}(a_i, a_j) = 1 \quad \forall i, j \in [1, n] \mid i \neq j$
- presi \( b_1, \ldots, b_n \in \mathbb{Z} \mid 0 \leq b_{1}<a_{1}, 0 \leq b_{2}<a_{2}, \ldots 0 \leq b_n \lt a_n\)
- $m := \textrm{mcm}(a_1, \ldots, a_n) = a_1 \cdot \ldots \cdot a_n$
- allora $\exists ! x \ (\bmod m)$ \( \left\{\begin{array}{c}x \equiv b_{1}\ \left(\bmod a_{1}\right) \\ \vdots \\ x \equiv b_{n}\ \left(\bmod a_{n}\right)\end{array}\right. \)
  - per il **lemma 1** $m = a_1 \cdot \ldots \cdot a_n$ poiché coprimi in ipotesi
  - per il **lemma 2** $m = \textrm{mcm}(a_1, \ldots, a_n) \implies \exists \phi : \mathbb{Z}_m \rightarrow \mathbb{Z}_ {a_1} \times \cdots \times \mathbb{Z}_{a_m}$ ben definita e iniettiva
  - \( \left|X_{1} \times \cdots \times X_{n}\right|=\left|X_{1}\right| \cdot\ldots\cdot\left|X_{n}\right| \implies\) \( \left|\mathbb{Z}_{a_{1}} \times \ldots \times \mathbb{Z}_{a_{n}}\right|=\left|\mathbb{Z}_{a_{1}}\right| \cdot\ldots\cdot\left|\mathbb{Z}_{a_{n}}\right| \)
    - \( \mathbb{Z}_n = \{[0],[1], \cdots,[n-1]\} \implies \left|\mathbb{Z}_{n}\right|=n\), quindi \(\left|\mathbb{Z}_{a_{1}}\right| \cdot\ldots\cdot\left|\mathbb{Z}_{a_{n}}\right|  = a_1 \cdot \ldots \cdot a_n = m = \left| \mathbb{Z}_m \right|\) per ragionamento analogo
  - \( |X|=|Y|<\infty \implies f: X \rightarrow Y \) iniettiva $\iff$ $f$ suriettiva
    - applicando questa osservazione, $\phi$ iniettiva $\implies \phi$ suriettiva, in quanto, per l'osservazione precedente, insieme di partenza e di arrivo di $\phi$ hanno la stessa cardinalità $\left| \mathbb{Z}_m \right|$
  - $\phi$ **suriettiva** $\implies$ $\exists x \mid x \ (\bmod m)$ è soluzione del sistema
    - \( \varphi(x \ (\bmod m))=\left(b_{1}\ \left( \bmod  a_{1}\right), \ldots, b_{n} \ (\bmod a_{n})\right) \), e poiché $\phi$ è suriettiva, allora ogni tupla di $n$ elementi dell'insieme di arrivo, che descrive un sistema come in ipotesi, ha una controimmagine $x \ (\bmod m)$, e $x \ (\bmod m)\in \mathbb{Z}_m$ per definizione, dunque **esiste sempre una soluzione**
  - $\phi$ **iniettiva** $\implies$ $\exists ! x \mid x \ (\bmod m)$ è soluzione del sistema
    - poiché $\phi$ è iniettiva, $x \ (\bmod m) \in \mathbb{Z}_m$ è unica, dunque **la soluzione è sempre unica**
